\documentclass[11pt]{jsarticle}%書体
\usepackage{ascmac}
\usepackage{amsmath}%数式
\usepackage[dvipdfmx]{graphics}%画像
\usepackage{upgreek}%ギリシャ文字立体
%\usepackage{name}

\title{タイトル}%タイトル
\author{著者}%著者
\date{日付}%日付

\begin{document}%本文はじめ
\maketitle%タイトル
\tableofcontents%目次
\listoffigures%図目次

\section{name}%セクション

\subsection{name}%サブセクション

\subsubsection{name}%サブサブセクション

\begin{figure}[h]%画像はじめ[ページ上部]
 \centering%中心揃え.
 %\includegraphics[clip,width=9cm]{figure}%[切り取り、サイズ指定]ファイル名
 \caption{caption.}%グラフタイトル.
 \label{label}%グラフ識別子
\end{figure}%画像おわり

\begin{equation}%数式はじめ
 \Delta E=\frac{\mathrm{\hbar}^{2}\uppi^{2}}{2mL^{2}}%\frac{a}{b}->a/b
 \label{eq_qw}%式識別子
\end{equation}%数式おわり

\begin{table}[ht]%表はじめ
 \centering
 \caption{実験1の実験条件.}
 \begin{tabular}{lc}\hline
  \multicolumn{2}{c}{実験条件} \\ \hline
  中心波長 & 435 nm            \\
  露光時間 & 0.1 秒            \\
  積算回数 & 1 回              \\
  試料温度 & 19 C    \\ \hline
 \end{tabular}
 \label{col_1}%識別子
\end{table}%表おわり

\begin{figure}[ht]%画像はじめ
 \centering%中央揃え
 \begin{tabular}{c}%横ならべ
  %1
  \begin{minipage}{0.5\hsize}%占有割合指定test
   \centering
   %\includegraphics[clip, width=4.5cm,height=3cm]{figure}
   \hspace{1.6cm} [1]サブタイトル%サブタイトル1
  \end{minipage}

  %2
  \begin{minipage}{0.33\hsize}
   \centering
   %\includegraphics[clip, width=4.5cm]{figure}
   \hspace{1.6cm} [2]サブタイトル%サブタイトル2
  \end{minipage}
 \end{tabular}%画像ならべおわり
 \caption{caption.}%グラフタイトル
 \label{label}%識別子

\end{figure}

\end{document}%本文おわり
